\documentclass{article}
\usepackage[dvipdfm]{color}
\usepackage[dvipdfm,a4paper,bookmarks=true,bookmarksopen=true,bookmarksnumbered=true,bookmarkstype=toc,colorlinks=false,linkcolor=white,linkbordercolor=0 0 0,pdftitle={FFTSS Library Version 3.0 User's Guide},pdfauthor={Akira NUKADA},baseurl={http://www.ssisc.org/fftss/fftssmanual-3.0-20071031.pdf},]{hyperref}
\linespread{1.2}
\oddsidemargin -1cm
\evensidemargin -1cm
\textwidth 7in
\topmargin -1.5cm
\textheight 9in
\begin{document}
\pagestyle{plain}
\hfill{\large Last Modified: 31/10/07}
\vspace{9cm}
\begin{center}
{\Huge \bf FFTSS Library Version 3.0\\User's Guide}
\end{center}
\vspace{8cm}
{\large Copyright (C) 2002-2007 The Scalable Software Infrastructure Project,\\
is supported by the ``Development of Software Infrastructure for Large Scale Scientific Simulation'' Team, CREST, JST. \href{http://www.ssisc.org/}{http://www.ssisc.org/}}

\pagestyle{empty}
\newpage
\tableofcontents
\newpage
\section{Introduction}
The FFTSS library is a Fast Fourier Transform (FFT) library that is distributed as
an open source software. The FFTSS library is one of the products of the
% the Scalable Software Infrastructure (SSI) project,
``Development of Software Infrastructure for Large Scale Scientific Simulation''
project, which is supported by
Core Research for Evolutional Science and Technology (CREST) of
the Japan Science and Technology Agency (JST).

The target of this library is to achieve high performance (high speed) in
various computing environments. The software package of this library contains
several types of FFT kernel subroutine. The library executes all of them 
 and selects the fastest subroutine for the target computing environment.

The interfaces of the library are
similar to those of the FFTW library version 3.x. This makes it easy
for FFTW users to use the FFTSS library. 
Porting to the FFTSS library requires a very slight modification
 of the source code.

\section{DFT}

The one-dimensional DFT of length $n$ in the FFTSS library 
actually computes\\
% the operations described below.
the forward transform\\
\[Y_k = \sum_{j=0}^n X_j\cdot e^{-2\pi jk\sqrt{-1}/n}\]
and the backward transform\\
\[Y_k = \sum_{j=0}^n X_j\cdot e^{2\pi jk\sqrt{-1}/n}.\]

$X$ is the array of input complex data in double precision,
and $Y$ is the output.
The results computed by the FFTSS library are {\bf not scaled} and
are compatible with the FFTW library.

The multi-dimensional transforms simply compute the 1-D transform along
each dimension of the array. 

\section{Installation}
The package of this library is distributed as source code.
Users need to build the library before using from application programs.

\subsection{UNIX and Compatible Systems}
To build the library software from the source package on UNIX or
compatible systems,
the following three steps are required:\\ 
1. Run `configure' script.\\
2. Run `make'.\\
3. Run `make install' (optional).\\

The configure script of this library accepts all generic flags.
 In addition, the flags listed below are also available.\\

\begin{tabular}{ll}
{\tt --without-simd}&Do not use SIMD instructions.\\
{\tt --without-asm}&Do not use assembly codes.\\
{\tt --with-bg}&Build for IBM Blue Gene system. (cross build)\\
{\tt --with-bg-compat}&Enable FFT kernels for Blue Gene in compatible mode.\\
{\tt --with-recommended}&Set recommended CC and CFLAGS variables.\\
{\tt --enable-openmp}&Enable OpenMP.\\
{\tt --enable-mpi}&Enable MPI.\\
\end{tabular}\\

You can also set CC and CFLAGS manually as follows.\\
\$ ./configure CC=gcc CFLAGS='-O3 -msse2'\\


\subsection{Microsoft Windows}
On Microsoft Windows systems, three methods are available 
for building the software.

\subsubsection{Visual Studio}
In the 'win32' folder of the package, a solution file `fftss.sln' 
is provided for Visual Studio 2003.NET. Open this file, and edit the 
compiler setting as you like.
If you have Visual Studio 2005 series, convert the solution file first.
Older versions of Visual Studio are not supported.

\subsubsection{Intel C/C++ Compiler}
In the `win32' folder of the package, batch scripts for Intel C/C++ Compiler
 are included. Run one of these scripts on the `Command Prompt' or 
corresponding `Build Environment' of the Intel C/C++ Compiler.
The working directory must be the `win32' folder.
If you want to change compiler options, edit the batch files directly.
%Microsoft Platform SDK must be installed on the system.
\\

\begin{tabular}{ll}
icl-x86.bat&For Intel IA-32 Architecture. (32bit)\\
icl-amd64&For Intel EM64T or AMD64 Architecture. (64bit)\\
icl-ia64&For Intel IA-64 Architecture. (64bit)\\
\end{tabular}

\subsubsection{MinGW environment}
In the MinGW environment, the configure script, as well as 
UNIX or compatible systems, is usable.

\section{Building Applications}

\subsection{UNIX and compatible systems}
In general, you need to
\begin{itemize}
\item{add ''-I/path/to/header/file" to CFLAGS}
\item{add ''-L/path/to/library/file -lfftss" to LDFLAGS}
\item{add ''-lpfftss" to LDFLAGS (for MPI only)}
\end{itemize}
to build your application with the FFTSS library.

\subsection{Visual Studio}
After building the entire binary of the FFTSS library using Visual Studio,
 a library file `{\bf fftss.lib}' is found in 
`./libfftss/Release'
 (or `./libfftss/Debug').
% A dll file `{\bf fftss.dll}' may also be found
%if the library is built as a dynamic library.
 The header file `fftss.h' is found in the `include' folder.
To use the FFTSS library with your application,
you should use these files by editing the properties of 
Visual Studio project files.

\section{Limitations}

In the current version, the FFTSS library only includes complex-to-complex,
 double precision (of floating-point numbers) routines.
The length of 1-D transforms must be a power of two.
In case of the multi-dimensional transform,
 the size of each dimension must be a power of two.

Since this library uses Stockham's auto-sort algorithm,
all of the FFT kernels included in this library perform
out-of-place transforms. Even if you request in-place transform
in FFTW style, the library allocates the buffer for out-of-place 
transform internally. 


\section{List of library functions}
\subsection{Memory Allocation}
\subsubsection{fftss\_malloc}
Syntax:\\
{\bf void *fftss\_malloc(long }\underline{size}{\bf );}\\

{\bf fftss\_malloc()} allocates \underline{size} bytes and returns a pointer to
the allocated memory. The address returned by the function is always 
aligned to a 16-byte boundary.

\subsubsection{fftss\_free}
Syntax:\\
{\bf void fftss\_free(void *}\underline{ptr}{\bf );}\\

{\bf fftss\_free()} frees the memory space pointed by \underline{ptr},
 which must have been returned by a previous call to {\bf fftss\_malloc()}.

\subsection{Creating Plans}

\subsubsection{fftss\_plan\_dft\_1d}
Syntax:\\
{\bf fftss\_plan fftss\_plan\_dft\_1d(long }\underline{n}
{\bf , double *}\underline{in}{\bf , double *}\underline{out}
{\bf , long }\underline{sign}{\bf , long }\underline{flags}
{\bf );}\\

{\bf fftss\_plan\_dft\_1d()} creates a plan for computing complex-to-complex double precision one-dimensional transforms of length 
\underline{n}. The real part of the $i$-th element of the input sequence must be stored in \underline{in}[i*2],
and the imaginary part must be stored in \underline{in}[i*2+1]. 

\subsubsection{fftss\_plan\_dft\_2d}
Syntax:\\
{\bf fftss\_plan fftss\_plan\_dft\_2d(long }\underline{nx}
{\bf , long }\underline{ny}{\bf , long }\underline{py}
{\bf , double *}\underline{in}{\bf , double *}\underline{out}
{\bf , long }\underline{sign}{\bf , long }\underline{flags}
{\bf );}\\

{\bf fftss\_plan\_dft\_2d()} creates a plan for computing complex-to-complex
double precision \underline{nx} by \underline{ny} two-dimensional transforms.
The real part of element(x,y) must be stored in 
\underline{in}[x*2+y*py*2],
and the imaginary part must be stored in \underline{in}[x*2+y*py*2+1]. 

\subsubsection{fftss\_plan\_dft\_3d}
Syntax:\\
{\bf fftss\_plan fftss\_plan\_dft\_3d(long }\underline{nx}
{\bf , long }\underline{ny}{\bf , long }\underline{nz}
{\bf , long }\underline{py}{\bf , long }\underline{pz}
{\bf , double *}\underline{in}{\bf , double *}\underline{out}
{\bf , long }\underline{sign}{\bf , long }\underline{flags}
{\bf );}\\

{\bf fftss\_plan\_dft\_3d()} creates a plan for computing complex-to-complex
double precision \underline{nx} by \underline{ny} by \underline{nz} three-dimensional transforms.
The real part of element(x,y,z) must be stored in 
\underline{in}[x*2+y*py*2+z*pz*2],
and the imaginary part must be stored in \underline{in}[x*2+y*py*2+z*pz*2+1]. 

\subsubsection{List of Flags}
 The following lists the available flags for creating plans
 for 1-D, 2-D, and 3-D transforms.

\begin{itemize}
\item{FFTSS\_VERBOSE}

This flag enables verbose mode.
In general, however, application users do not require verbose mode.
Using this flag, the name of the selected FFT kernel is shown in standard output.

\item{FFTSS\_MEASURE}

This is the default setting. In creating plans, the library executes all FFT kernels
 available in the computing environment and select the best kernel.

\item{FFTSS\_ESTIMATE}

The library estimates the best FFT kernel for the computing environment
 without any executions.

\item{FFTSS\_PATIENT}

Same as FFTSS\_MEASURE.

\item{FFTSS\_EXHAUSTIVE}

Same as FFTSS\_MEASURE.

\item{FFTSS\_NO\_SIMD}

This flag disables the use of SIMD (or SIMOMD) instructions.
Try this flag if you have some trouble with the FFT kernels 
using SIMD instructions.

\item{FFTSS\_UNALIGNED}

Specify whether the input array is not aligned to a 16-byte boundary.
The alignments of the input and the output buffers are checked
when creating plans. If they are not aligned, this flag is added automatically.
Therefore, this flag is usually not necessary.
This flag must be specified only if all of the conditions listed below are
satisfied.
\begin{enumerate}
\item{Aligned buffers are given when creating plans.}
\item{Unaligned buffers are used with {\bf fftss\_execute\_dft()}.}
\item{The selected FFT kernel requires alignment.}
\end{enumerate}

Since the input and output buffers should be aligned
 from the aspect of performance, the use of {\bf fftss\_malloc()} is strongly 
recommended for memory allocation.

\item{FFTSS\_DESTROY\_INPUT}

This flag allows destruction of data in the input buffer. (default)

\item{FFTSS\_PRESERVE\_INPUT}

The data in the input buffer is preserved, and working space
will be allocated by the library function for out-of-place transforms.

\item{FFTSS\_INOUT}

When this flag is specified, the results of the transforms are returned to the input buffer \underline{in}. The output buffer \underline{out} must also be specified
because it will be used for working space.


\end{itemize}

\subsection{Executing Plans}

\subsubsection{fftss\_execute}
Syntax:\\
{\bf void fftss\_execute(fftss\_plan }\underline{p}
{\bf );}\\

{\bf fftss\_execute()} executes a plan \underline{p}.

\subsubsection{fftss\_execute\_dft}
Syntax:\\
{\bf void fftss\_execute\_dft(fftss\_plan }\underline{p}
{\bf , double *}\underline{in}{\bf , double *}\underline{out}{\bf );}\\

{\bf fftss\_execute()} executes a plan \underline{p}.
The input buffer \underline{in} and output buffer \underline{out} are
used than the value specified when creating the plan \underline{p}.

\subsection{Destroying Plans}

\subsubsection{fftss\_destroy\_plan}
Syntax:\\
{\bf void fftss\_destroy\_plan(fftss\_plan }\underline{p}
{\bf );}\\

{\bf fftss\_destroy\_plan()} deallocates the plan \underline{p}.

\subsection{Timer}

\subsubsection{fftss\_get\_wtime}
Syntax:\\
{\bf double fftss\_get\_wtime(void);}\\

{\bf fftss\_get\_wtime()} returns the current timestamp in seconds.

\subsection{Multi-Threading}

\subsubsection{fftss\_init\_threads}
Syntax:\\
{\bf int fftss\_init\_threads(void);}\\

The {\bf fftss\_init\_threads()} function does nothing and exists only 
for reasons of compatibility with FFTW3.

\subsubsection{fftss\_plan\_with\_nthreads}
Syntax:\\
{\bf void fftss\_plan\_with\_nthreads(int }\underline{nthreads}{\bf );}\\

{\bf fftss\_plan\_with\_nthreads()} sets the number of threads 
 used for computation. Since the FFTSS library supports only
 parallelization with OpenMP, this function simply 
sets the number of OpenMP threads using {\bf omp\_set\_num\_threads()}.

\subsubsection{fftss\_cleanup\_threads}
Syntax:\\
{\bf  void fftss\_cleanup\_threads(void);}\\

The {\bf fftss\_cleanup\_threads()} function does nothing and exists only for reasons of compatibility with FFTW3.

\subsection{MPI}
The FFTSS library version 3 includes MPI interfaces with
prefix 'pfftss\_'.
\subsubsection{pfftss\_plan\_dft\_2d}
Syntax:\\
{\bf pfftss\_plan pfftss\_plan\_dft\_2d(long }\underline{nx}{\bf, long }\underline{ny}{\bf, long }\underline{py}{\bf, long }\underline{oy}{\bf, long }\underline{ly}{\bf, double *}\underline{inout}{\bf, long }\underline{sign}{\bf, long }\underline{flags}{\bf, MPI\_Comm }\underline{comm}{\bf );}\\

{\bf pfftss\_plan\_dft\_2d()} creates a plan for computing complex-to-complex
double precision \underline{nx} by \underline{ny} two-dimensional transforms
for distributed parallel computers with MPI library.
The two-dimensional array is used for both input and output, and therefore
the input array \underline{inout} is always overwritten. Each node has 
\underline{ly} rows from the \underline{oy}-th row of 
the \underline{nx} by \underline{ny} array, and the nodes are stored in a
\underline{py} by \underline{ly} array (double \underline{inout}[2*\underline{py}*\underline{ly}]). The data must be block-row distributed between nodes in the order of their MPI rank in \underline{comm}.

\underline{nx} must be not less than the number of nodes, and
\underline{ly} must be non-zero on all nodes.
This function internally allocates memory for two arrays 
of approximately the same size as the input array.


\subsubsection{pfftss\_execute}
Syntax:\\
{\bf void pfftss\_execute(pfftss\_plan }\underline{p}{\bf);}\\

{\bf pfftss\_execute()} executes a plan \underline{p} using MPI library.


\subsubsection{pfftss\_execute\_dft}
Syntax:\\
{\bf void pfftss\_execute\_dft(pfftss\_plan }\underline{p}{\bf, double *}\underline{inout}{\bf);}\\

{\bf pfftss\_execute\_dft()} executes a plan \underline{p} using MPI library.
\underline{inout} is used for the input and output array rather than the
value specified when creating the plan.

\subsubsection{pfftss\_destroy\_plan}
Syntax:\\
{\bf void pfftss\_destroy\_plan(pfftss\_plan }\underline{p}{\bf);}\\

{\bf pfftss\_destroy\_plan()} destroys a plan \underline{p}, and
deallocates all memory areas allocated for the plan.

\section{Multi-Threading}
The current version of the FFTSS library supports multi-threading with OpenMP.
To build a library for multi-threading, an OpenMP compiler is required,
and the compiler options required for OpenMP support must be added.

For example, a build with Intel C/C++ Compiler is described below.
The option '-openmp' enables support for OpenMP, and
'-xP' enables support for Intel SSE3 instructions.\\
\$ ./configure CC=icc CFLAGS='-O3 -openmp -xP'

To specify the number of threads, the environment variable 
'OMP\_NUM\_THREADS' is used, as well as other OpenMP applications.
If this variable is not set, the number of threads depends on the
computing environment and typically is equal to one or the 
number of processors. In addition, you can set the number of threads with the 
{\bf omp\_set\_num\_threads()} function. We provide a library function `fftss\_plan\_with\_nthreads()' for compatibility with the FFTW library. This library function simply calls the {\bf omp\_set\_num\_threads()} function.

The number of threads must be set before creating plans
because the work space for each thread is allocated. If you need to use a variable number of threads, the maximum number of threads must be set when creating plans.

\begin{verbatim}
   max_threads = omp_get_num_procs();
   fftss_plan_with_nthreads(max_threads);
   plan = fftss_plan_dft_2d(nx, ny, py, vin, vout, 
      FFTSS_FORWARD, FFTSS_MEASURE);

   {  /*  Initialize arrays.  */  }

   for (nthreads = 1; nthreads <= max_threads; nthreads ++) {
      fftss_plan_with_nthreads(nthreads);
      t = fftss_get_wtime();
      fftss_execute(plan);
      t = fftss_get_wtime() - t;
      printf("%lf sec with %d thread(s).\n", t, nthreads);
   }
\end{verbatim}

You can use OpenMP multi-threading with the MPI library
if you specified both '--enable-mpi' and '--enable-openmp' options.
The library is then built for MPI+OpenMP hybrid parallelization.

\section{Compatibility with the FFTW library}
The interfaces of this library are similar to those of the FFTW library version 3.

The FFTW users can use the FFTSS library in compatible mode. Application programs written for the FFTW library include header file `fftw3.h'. To use the FFTSS library, users need only change the name of the header file to `fftw3compat.h'.

In the header file `fftw3compat.h', a number of macros are defined in order to convert the source code for the FFTW library, and the header file `fftss.h' is also included in this file. 

In the current version, only a subset of the FFTW3 library is available, which is defined or declared in the header file `fftw3compat.h'. The MPI version has no compatibility.

\subsection{Compatible Functions}
\begin{itemize}
\item{\bf fftw\_malloc()}
\item{\bf fftw\_free()}
\item{\bf fftw\_plan\_dft\_1d()}
\item{\bf fftw\_plan\_dft\_2d()}
\item{\bf fftw\_plan\_dft\_3d()}
\item{\bf fftw\_execute()}
\item{\bf fftw\_execute\_dft()}
\item{\bf fftw\_destroy\_plan()}
\item{\bf fftw\_init\_threads()}
\item{\bf fftw\_plan\_with\_nthreads()}
\item{\bf fftw\_cleanup\_threads()}
\end{itemize}

\subsection{Compatible Flags}
\begin{itemize}
\item{\bf FFTW\_MEASURE}
\item{\bf FFTW\_ESTIMATE}
\item{\bf FFTW\_PATIENT}
\item{\bf FFTW\_EXHAUSTIVE}
\item{\bf FFTW\_NO\_SIMD}
\item{\bf FFTW\_PRESERVE\_INPUT}
\item{\bf FFTW\_DESTROY\_INPUT}
\item{\bf FFTW\_FORWARD}
\item{\bf FFTW\_BACKWARD}
\end{itemize}

\section{List of FFT Kernels}
In this section, the available FFT kernels are listed.
The name of the selected kernel 
can be displayed by using the {\bf FFTSS\_VERBOSE} flag.
\begin{itemize}
\item{normal}\\
Normal implementation.
\item{FMA}\\
An implementation of FFT kernels optimized for Fused Multiply-Add (FMA)
 instructions.
\item{SSE2 (1)}\\
An implementation with Intel SSE2 instructions.
\item{SSE2 (2)}\\
An implementation with Intel SSE2 instructions. (UNPCKHPD/UNPCKLPD)
\item{SSE3}\\
An implementation with Intel SSE3 instructions. (ADDSUBPD)
\item{SSE3 (H)}\\
An implementation with Intel SSE3 instructions. (HADDPD/HSUBPD)
\item{C99 Complex}\\
An implementation using the C99 Complex data type.
\item{Blue Gene}\\
An implementation for IBM Blue Gene.
\item{Blue Gene (PL)}\\
An implementation for IBM Blue Gene (Software pipelined).
\item{Blue Gene asm}\\
An implementation in assembly language for IBM Blue Gene.
\item{IA-64 asm}\\
An implementation in assembly language for the Intel IA-64 architecture.
\end{itemize}

\section{Tested Platforms}
The FFTSS library has been tested on the following platforms.\\
\vspace{1cm}
\begin{tabular}{|l|l|l|}
\hline
Processor&OS&Compiler\\
\hline
UltraSPARC III&Sun Solaris 9&Sun ONE Studio 11\\
\hline
Itanium 2&Linux&Intel C/C++ Compiler 9.1, gcc 4.0.1\\
\hline
PowerPC G5&Mac OS X 10.4&IBM XL C Compiler 6.0, gcc 4.0\\
\hline
POWER5&Linux&IBM XL C Compiler 7.0, gcc 4.0.1\\
\hline
POWER4&AIX&IBM XL C Compiler 6.0\\
\hline
PA-RISC&HP-UX 11&Bundled C Compiler\\
\hline
PPC440FP2&Blue Gene CNK&IBM XL C Compiler 7.0/8.0\\
\hline
Opteron&Linux&gcc 3.3.3, gcc 4.0.1\\
\hline
Pentium 4&Solaris 9 (IA-32)&Sun ONE Studio 11, gcc 4.0.1\\
\hline
Xeon&Linux&Intel C/C++ Compiler 8.1/9.0/9.1, gcc\\
\hline
IA-32&Windows XP SP2&Visual Studio.NET 2003\\
\hline
IA-32&Windows XP SP2&Visual Studio 2005\\
\hline
IA-32&Windows XP SP2&Intel C/C++ Compiler 9.1\\
\hline
x64&Windows XP, 2003&Visual Studio.NET 2003\\
\hline
x64&Windows XP, 2003&Intel C/C++ Compiler for EM64T 9.1\\
\hline
\end{tabular}

\end{document}

